\subsection{Traffic forecasting}
\label{sub:ts_forecasting}
Capturing data about traffic in a highway worth not only for describing its current situation, but also for forecasting future conditions. In this sense, counting the number of cars that have passed for a given point in a given time can be considered as a time series, allowing the use of automatic tools to perform forecasting.

For this work, 4 methods (3 widely used time series forecasters, and 1 control method) have been used to, automatically, predict the number of cars that will pass in the next period. These methods are included in the forecast package ~\cite{Hyndman08automatictime} of the R program~\cite{R:Bloomfield:2014}, and are the following:

\begin{itemize}
\item {\em Exponential smoothing state space model (ETS)~\cite{ETS:2008}}: Represent a set of methods that decompose the time-series in three different characteristics: error, trend and seasonal, modelling each one with a given equation. The forecasted values are obtained joining the previous equations by addition or multiplication. The corresponding function in the forecast package builds the different models included in the ETS set, and select the one with the best according to the Akaike's Information Criterion (AIC)~\cite{Akaike1973}.
\item {\em ARIMA~\cite{BoxJenk}: This well known method, due to Box and Jenkins } integrates autoregressive (AR) and moving average (MA) models in a three-stage iterative cycle. The phases of every cycle consist of: identifying the time series, estimating of the model's parameters, and verifying the built model. Briefly, every model is defined as a sum of $p+q$ terms. The first $p$  terms are defined by $p$ past values, any of them mulitplied by a coefficient; while the last $q$ terms represent the moving averages, also multiplied by their own coefficients.
\item {\em Theta~\cite{Assima2000}}: %%This paper presents a new univariate forecasting method. The method is based on the concept of modifying the local curvature of the time-series through a coefficient ‘Theta’ (the Greek letter θ), that is applied directly to the second differences of the data. The resulting series that are created maintain the mean and the slope of the original data but not their curvatures. These new time series are named Theta-lines. Their primary qualitative characteristic is the improvement of the approximation of the long-term behavior of the data or the augmentation of the short-term features, depending on the value of the Theta coefficient. The proposed method decomposes the original time series into two or more different Theta-lines. These are extrapolated separately and the subsequent forecasts are combined. The simple combination of two Theta-lines, the Theta=0 (straight line) and Theta=2 (double local curves) was adopted in order to produce forecasts for the 3003 series of the M3 competition. The method performed well, particularly for monthly series and for microeconomic data.
\item {\em Mean}: This is the control method. The future values are computed as the mean of the previous ones. Despite being one of the simplest forecaster methods, tends to achieve quite good results for very short horizons.
\end{itemize}

The data used for the experiments is the ones provided by DGT (see table ~\ref{tab:nodosDGT}), except node $1040$, which did not record data for many periods. For every node, the number of cars detected have been accumulated in periods of 15 minutes. This has allowed to create a database of 5 files, any of them with $2016$ rows, starting on 12-Oct-2015, 00:00 hours, and finishing on 01-Nov-2015, 23:45 hours, i.e., three complet weeks. 

In order to 




%% ---------------------- REFERENCIAS que aún no he copiado a mobility.bib

